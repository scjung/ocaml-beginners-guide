\section{통합개발환경}\label{sec:ide}

이 절에서는 편리한 \OCAML{} 개발을 위한 \KOEN{통합개발환경}{integrated
  development environment} (이하 IDE)을 구축하는 방법을
소개합니다. 공식적으로는 \OCAML{} 도구는 전부 \KOEN{명령줄}{command-line}
환경만을 지원합니다. 하지만 많은 오픈소스 개발자의 노력에 힘입어 이제는
검은 화면 터미널을 떠나 어느정도 괜찮은 IDE 위에서 개발할 수 있게 되었습니다.

여기서 소개하는 IDE는 크게 두 가지로 나뉩니다. \textsf{Java} 개발 환경으로
유명한 \ECLIPSE{}에 \OCAML{} 플러그인을 장착한 것과, 전통이 있는 만능 편집기
\EMACS{}에 \OCAML{} 패키지를 장착한 것. 자신이 편한 것을 선택해서
진행합시다. 만일 \EMACS{}가 낯설다면 그나마 \ECLIPSE{}가 \EMACS{} 보다 훨씬
눈에 익숙할 테니 \ECLIPSE{} 쪽을 사용하는 것이 좋을 것입니다.

\subsection{\ECLIPSE{}와 \OCAIDE{}}

\TODO{\URL{http://www.algo-prog.info/ocaide}}

\subsection{\EMACS{}와 \TYPEREX{}}

\TODO{\URL{http://www.typerex.org}}



%%% Local Variables: 
%%% mode: latex
%%% TeX-master: "master"
%%% End: 
