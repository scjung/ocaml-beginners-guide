\section{머리말}\label{sec:intro}

\TODO{머리말}

\paragraph{\OCAML{} 버전} 이후 문서 내 \OCAML{}은 별다른 언급이 없는 한 최신
버전, \OCAML{} 4.00.0을 의미합니다. \OCAML{} 버전은 오픈소스에서 주로
사용되는 방식을 따라 메이저버전.마이너버전.버그수정버전 순으로
매겨집니다. 자신이 사용하는 \OCAML{}의 버그수정버전이 다르더라도 대부분의
내용에는 문제가 없을 것입니다. 하지만 메이저버전이나 마이너버전이 다른 경우는
몇가지 문제가 있을 수 있습니다. 되도록 \OCAML{} 3.12 버전 이상의 최신 버전을
사용하세요.

\paragraph{표기법} 이후 문서에서 쓰이는 표기법에 대해 살펴봅시다. 우선
\KOEN{명령줄}{command-line}에서 입력해야 할 명령은 다음과 같은 형태로
표기합니다.

\begin{lstlisting}
$ ~ls~
foo   foo.ml
...
\end{lstlisting}

맨 왼쪽 숫자는 편의를 위한 줄번호이며 맨 앞의 \texttt{\$} %$
는 명령 프롬프트를 나타냅니다. 입력해야 할 명령은 프롬프트 다음에 나오는
\textbf{굵은} 글씨로 쓰여진 내용입니다.
즉 앞의 예제에서 입력해야 할 명령은 \texttt{ls} 입니다.
그 다음 줄부터는 명령을 수행하면 나오는 결과를 나타냅니다. 결과 중
\texttt{...}은 나머지 결과가 생략되었음을 의미합니다.

\paragraph{이후 내용} 문서는 다음과 같이 이뤄집니다. 먼저 다음 1절에서는
\OCAML{} 시스템의 구조와 사용에 앞서 알아야 할 기초 지식에 대해 설명합니다.
그리고 나서 2절에서는 각 플랫폼 별로 \OCAML{}을 설치하는 법, 3절은 \OCAML{}
프로그래밍을 도와주는 통합개발환경을 어떻게 구성하는지 알아봅니다. 2절과 3절을
마치면 왠만한 \OCAML{} 프로그램은 무리없이 개발할 수 있는 그럴싸한 환경을 갖출
수 있습니다. 마지막 4절에서는 \OCAML{} 프로그래밍을 배우는 여정을 떠나는 데
도움이 될 여러 참고 자료들을 소개합니다.

\paragraph{웹사이트} 이 문서는 다음 \textsf{Github} 사이트에 공개됩니다.

\begin{center}
\URL{https://github.com/scjung/ocaml-beginners-guide}
\end{center}

문제점, 건의사항 등 어떠한 제안도 환영합니다. \textsf{Github} 사이트 혹은 다음
이메일을 이용해 주세요.

\begin{center}
\href{mailto:scjung@pav.hanyang.ac.kr}{\texttt{scjung@pav.hanyang.ac.kr}}
\end{center}


%%% Local Variables: 
%%% mode: latex
%%% TeX-master: "master"
%%% End: 
