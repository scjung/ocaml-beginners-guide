\section{\OCAML{}이란}\label{sec:ocaml}

\TODO{OCaml이란}

\subsection{\FINDLIB{}}

\TODO{findlib}

\subsection{\GODI{}}

대부분의 \OCAML{} 라이브러리는 오픈소스로 개발되며 개발자는 바이너리 없이
소스만을 제공합니다. 따라서 사용자는 소스를 내려받은 후 직접 컴파일을 해야
합니다. 작은 라이브러리는 대부분 내려받고, 설정하고, 컴파일하고, 설치하는
순으로 간단하게 설치할 수 있지요. 하지만 설치하려는 라이브러리가 다른
\OCAML{} 라이브러리에 의존하는 경우에는 일일히 라이브러리를 찾아서 설치
해줘야 하는 불편이 따릅니다. 더 심한 경우에는 라이브러리 버그로 인해 특별히
제작된 패치 없이는 컴파일 불가능한 경우도 생기죠.

\GODI{}는 Gerd Stolpmann이 개발한 \OCAML{} 라이브러리 관리
시스템입니다. \GODI{}는 사용자가 원하는 라이브러리를 내려받고, 설정하고,
패치를 적용하여 설치하는 일련의 과정을 자동으로 수행합니다. 또한 \GODI{}
시스템은 \GODI{}로 설치한 라이브러리 버전을 세심히 관리해서, 만일 새 버전이
나왔다면 자동으로 업데이트를 수행해주기도 합니다. \UBUNTU{}의
\texttt{apt-get}이나 \MAC{}의 \texttt{macport}를 아시나요? 그러한 시스템이
\OCAML{} 전용으로 설계되었다고 보면 됩니다. \OCAML{} 개발자에게는 여러모로
편리한 시스템입니다.

이후 \LINUX{}, \MAC{} 환경 하에서 설치법을 알아볼 때 \GODI{} 시스템을 설치하는
방법도 설명할 것입니다. \GODI{}에 대한 자세한 정보는 다음 웹사이트를 참고하세요.

\begin{center}
  \URL{http://godi.camlcity.org/godi}
\end{center}

%%% Local Variables: 
%%% mode: latex
%%% TeX-master: "master"
%%% End: 
